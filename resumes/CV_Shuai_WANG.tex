%-----------------------------------------------------------------------------
%	PACKAGES AND OTHER DOCUMENT CONFIGURATIONS
%------------------------------------------------------------------------------
\documentclass[12pt,a4paper,sans]{moderncv} % Font sizes: 10, 11, or 12;
%paper sizes: a4paper, letterpaper, a5paper, legalpaper, executivepaper or
%landscape; font families: sans or roman

\moderncvstyle{banking} % CV theme - options include: 'casual' (default),
%'classic', 'oldstyle' and 'banking'
\moderncvcolor{black} % CV color - options include: 'blue' (default), 'orange',
%'green', 'red', 'purple', 'grey' and 'black'

\usepackage{lipsum} % Used for inserting dummy 'Lorem ipsum' text into the
%template
\usepackage{verbatim}
\usepackage[scale=0.9]{geometry} % Reduce document margins
%\setlength{\hintscolumnwidth}{3cm} % Uncomment to change the width of the
%dates column
%\setlength{\makecvtitlenamewidth}{10cm} % For the 'classic' style, uncomment
%to adjust the width of the space allocated to your name
%------------------------------------------------------------------------------
%	NAME AND CONTACT INFORMATION SECTION
%------------------------------------------------------------------------------

\firstname{Shuai} % Your first name
\familyname{Wang} % Your last name

\title{Curriculum Vitae}
\address{Cincinnati, Ohio 45230}{}
\mobile{(937)-654-0806}
% \email{wang.172@wright.edu}
\email{vanstark88 at gmail.com}
\homepage{shuaiwang88.github.io}

%------------------------------------------------------------------------------

\begin{document}
\setlength{\parskip}{2em}
\makecvtitle % Print the CV title
%
% \begin{center}{
% \textbf{I build end-to-end data science and operations research project.}
%    } 
% \end{center}


%------------------------------------------------------------------------------
%	EDUCATION SECTION
%------------------------------------------------------------------------------

\section{Education}
\cventry{2011--2017}{Ph.D. in Engineering Program, Industrial and Human
System}
{Wright State University}{Dayton, Ohio}{}{}  
% Arguments not required can be left empty
%\cventry{2011--2012}{Master in Engineering and Entrepreneurship
%Program}{Wright State University}{Dayton, Ohio}{UNFINISHED}{}
\cventry{2007--2011}{Bachelor of Management, Supply Chain Management}{Dalian Jiaotong University}{Dalian, China}{}{}
%\vspace*{3mm} %this will fix vertical spacing for \moderncvstyle{classic}, but
%create vertical spacing issues for other \moderncvstyle{xyz}.

%-----------

%------------------------------------------------------------------------------
%	WORK EXPERIENCE SECTION
%------------------------------------------------------------------------------
%\section{}
\section{Computer Skills} 
\cvitem{Language}{Python, R, Julia; SQL}
\cvitem{Machine Learning}{regression, classification, clustering; xgboost, lightgbm, adaboost, random forest, etc} 
\cvitem{AutoML}{Datarobot, H2O, Knime; Pycaret, GlutonTS}
\cvitem{Optimization Modeling}{AIMMS, Pyomo, Minizinc, JuMP, OPL; Cplex, Gurobi, CBC, HiGHS }
\cvitem{Heuristic}{Simulated Annealing, Genetic Algorithm, Tabu Search, Golden Section Search, etc.}
\cvitem{Time Series}{Anomaly detection: KATS; Forecasting: ARIMA, Prophet, RNN}
\cvitem{Discrete Event Simulation}{Arena, AnyLogic, Sigma}
\cvitem{Database}{Snowflake, Redshift, Clickhouse, Postgres, MySql; dbt}
\cvitem{Cloud}{AWS Lambda, Sagemaker, Snowpark, Spark, Airflow, API Gateway, EC2}
\cvitem{Visualization}{Shiny, Looker, Superset, Tableau, Streamlit, Dash}


\section{Experience}
\cventry{2023.6--Current} {Senior Data Scientist, Full time}{\textsc{Pacific Life Insurance}}{Newport Beach, CA}{}{}
Leading data science projects in the areas of math optimization, forecasting; operations, customer service, call center. Mentoring junior data scientists.
\begin{itemize}
\item Developing various time series forecasting, by using AutoML, for processing task volume for annuity and life products.
\item Creating a MILP based skill-based routing task assignment system, by using HiGHS solver via Pyomo, to optimize tasks dispatching by considering specialist competence, task priority, time constraints, etc.
\item Leading the data science development for the Medallia VOC(Voice of Customer) for call center in the areas of topics, sentimental, trend.
\item Implementing virtual queue, call back policy to call center.
\item Re-creating mortality table(life table) using math optimization. 

\end{itemize}


\bigskip
 
\cventry{2021.4--2023.5} {Principal Data Scientist, Full time}{\textsc{Thomasnet}}{New York, NY}{}{
Reporting to head of data and AI. I started several 0 to 1 data initiatives at Thomasnet including:}
\begin{itemize}
\item \textbf{Internal facing:}
    \begin{enumerate}
    \item Applying AutoML to models like customer churn, bot detection, and sales forecasting.
    \item Recommending products portfolio to suppliers to maximize ROI using integer programming optimization.
    \item  Creating data cleaning, extraction, normalization python library for all the enterprise data ingestion pipeline.
    \item Introducing new tools like Superset, Datahub, and Amundsen to the enterprise for better data governance and visualization.
    \end{enumerate}
\item \textbf{External facing:}
    \begin{enumerate}
    \item Creating time series anomaly detection (signals) of sourcing activities for institutional investors.
    \item Creating sourcing trend and stock price/revenue correlation for institutional investors.
    \item Working on TMX index to track supply chain status in the USA. TMX is used by researchers from Carnegie Mellon University for Congressional hearing on supply chain, and University of Hong Kong for COVID19 related studies.
    \item Creating APIs and charts for the Thomasnet's alternative data platform for institutional investors with Datawheel.
    \end{enumerate}
\end{itemize}
 \bigskip

 
%\cventry{2022.11.9--Current} {Founder}
%{\textsc{AutoEncoder LLC}}{Cincinnati, OH}{}{}
%My own consulting practice.
%The clients include: 
%\begin{itemize}
%    \item McKinsey \& Company: Pricing and promotion planning for retail.
%    \item SightLine: College enrollment analysis. 
%\end{itemize}



\bigskip
\cventry{2017.9--2021.4}{Lead Operations Research Scientist, Contract}
{\textsc{Kroger/84.51}}{Cincinnati, OH}{}{
I transferred with Kroger's R\&D team, to Kroger's subsidiary data analytical company 84.51 in February 2018.}
\begin{itemize}
\item \textbf{Order Forecast and Picking Staff Daily Scheduling System Optimization}: 
    The daily staff scheduling system is built to optimize the number of staff required to picking 
    orders at each hour.
    Time-series based machine learning model is implemented to predict the orders. 
    An optimization model using CBC/Cplex solver is created to minimize labor cost. 
    The project saved about 20\% to 30\% labor cost (\$200 to \$300 million) than the previous implementation. 
    This project was escalated as one of top priority projects responding to COVID-19 at Kroger.
\item \textbf{In-store Inventory Control and Restock Optimization}:
The project is to transform the in-store inventory into scientific based management system, to increase productivity:
1. I build the inventory replenishment routing model, to reduce travel distance, using heuristics based on traveling salesman problem. A novel distance metric was proposed to match different store layouts. 
2. A comprehensive staff productivity graph was created to track the scanning activity, idle time, travel time.
3. Heuristic-based restock strategy was created to alert the restock point based on BOH. 


\end{itemize}



\bigskip
%\bigskip

\cventry{2012--2017}{Operations Research Consultant at 
Kroger}{\textsc{Large Scale Optimization Lab(Lab 249) from WSU partner with Kroger}}{Cincinnati, OH}{}
{} 
{}
%Role: make mathematical model, implement 
%core algorithm
% \newline{}
% \textbf{Key modeler for selected projects}:

\begin{itemize}
\item \textbf{Product Promotion Planning and Assortment Optimization}: A MILP optimization model (Collaborative Category Optimization) was
jointly developed with \textsc{A.T. Kearney} using AIMMS to build promotion planning and
assortment selection to maximize the adjusted gross revenue.
The system is implemented in 2014 and saves about 3\% to 5\% (\$2 to \$5 billion) of the total revenue annually.
%\newline{}
\item \textbf{Forecast and Optimization for the Little Clinics and Nurse Scheduling}: The problem is
to accurately forecast the number of patients by type to each clinic.
Sophisticated forecast models that utilize inputs from time, weather, social
media data like Google trend queries are used to predict the number of
visits. These forecasts serve as input to calculate the inventory necessary for
each type of sickness, and the allocation of medical personnel and their
shifts. The overall goal is to improve customer service and increase the number
of clinics from 136 to 500 in three to five years.
%\newline{}
\item \textbf{High Value Pharmacy Product Local Inventory Transfer}:  The problem is to ship 
medical drugs from stores with excess to stores with needs so as to: 1) reduce
potential obsolesces in inventory; 2) better position drugs to meet customer
demand; 3) to aggregate the shipment in such that transportation costs are
reduced. The model is piloted in 121 stores and is expected
to produce \$30 to \$50 million savings as well as reduction in
out-of-stock.

 \item \textbf{Periodic Vehicle Routing}: Each Kroger store has demand that
 fluctuates within a week in various categories, such as frozen, fresh, and
 grocery. Determining the frequency of visit to each store for each category is
 a complex periodic vehicle routing problem. Sophisticated optimization models
 have been built to determine the time of visits for each stores;  these
 optimized visit frequencies are expected to result in a reduction of 10\%
 (equivalent to \$150 million) in transportation cost.
 \item \textbf{Pharmacy Department Register Simulation}: Each store has a
 different volume in terms of patients visit and service time. The simulation
 model, based on queueing theory, was created to adjust the number of registers to be installed in
 each store, as a function of service time and customer volume.


\item \textbf{Call Center Staff Scheduling}:
I developed a call center staff scheduling model using genetic algorithm, to minimize under-coverage rate.
\end{itemize} 
 
\bigskip

\cventry{2011--2017} {Graduate Research and Teaching
Assistant}{\textsc{Wright State University}}{Dayton, OH}{}
{}
\begin{itemize}
\item \textbf{PhD Dissertation: Data mining techniques and mathematical models for the optimal scholarship allocation problem for a state university.}
The research uses classification algorithms to find
matriculation and graduation rate by varying scholarships. Then the
optimization model was developed to optimize revenue under budget, and fairness constraints.
This research has prompted the university wide scholarship redesign, the APS
calculator, see \href{http://www.wright.edu/raider-connect/financial-aid/first-year-scholarships}{Website}. 
This project has resulted a 11\% (2014) and 13.9\% (2015) increase in direct admit
students, which translates into a 5 to 10 million dollars of revenue increase
for WSU annually.
\item \textbf{Teaching Assistant:}
1. Intro of Data Mining and Applications. 
2. Intro of Operations Research Models.
\item \textbf{Dissertation Committee:}
    \begin{enumerate}
\item Xinhui Zhang(Supervisor): Director of Data Science at Coupang. Three times INFORMS Franz Edelman Award finalist.
\item Pratik Parikh: Chair of Dept of Industrial Engineering, University of Louisville.
\item Nan Kong: Purdue University. 
\item Caroline Cao: Director of Engineering Innovation \& Medical Simulation, University of Illinois Urbana-Champaign.
\item Subhashini Ganapathy: Dean of the College of Graduate Studies, Wright State University.
\end{enumerate}
 \end{itemize}
 
\bigskip

\cventry{}{Operations Research/Data Science Consultant}
{\textsc{Pro Bono}}{}{}{}{
\begin{itemize}
\item \textbf{Cincinnati Public School Bus Routing and Bell Time Optimization}: I wrote the core optimization algorithm using meta-heuristics, Julia with CBC/Gurobi solver. The result is comparable to the MIT's awarded solution for the Boston public school system.
\item \textbf{NYC Dog Care Stores Weekly Staff Scheduling Optimization}: I helped a dog 
care store with 5 locations to create a weekly staff scheduling system with various 
constraints such as: staff schedule preference, locations preference, demand coverage, and cross-skills satisfaction. 

\end{itemize}


}






%\newpage

%------------------------------------------------
%\cventry{2011.4--2011.8} {Marketing Intern}{\textsc{China Tobacco}}{Beijing,
%China}{}{Conducted intensive marketing analysis for The British American
%Tobacco Co. to broad the Chinese sales channel.}

%\cventry{09.6--09.8,08.6-08.8}{Inventory Management Intern, Business
%Management Intern}{\textsc{The River Seine Food Management }}{Shanghai,
%China}{}{Using ERP to track inventory and reverse logistics. Developed an
%algorithm to re-align the layout of each restaurant according to the daily
%reservation.}


%	AWARDS SECTION

%\begin{comment}
% \section{Awards and Other Information}
% \cvitem{2016}{Last year Ph.D. Assistantships, \textit{College of Engineering
% and Computer Science,Wright State University} }
% \cvitem{2012}{Ph.D. in Engineering Program University Assistantships,
% \textit{College of Engineering and Computer Science,Wright State University} }
% \cvitem{2011}{Provost Enrollment Scholarship, \textit{Wright State University
% Provost Office} }
%\cvitem{2011}{Student International Travel Grant, \textit{Dalian Jiaotong
%University} }
% \cvitem{Citizenship} {P.R.China. USA permanent resident}
%\end{comment}



%	COMPUTER SKILLS SECTION
%------------------------------------------------------------------------------


\comment{





\section{Languages}

\cvitemwithcomment{Chinese}{Mother tongue}{}
\cvitemwithcomment{English}{Fluent}{IELTS:7}
%------------------------------------------------------------------------------
%	INTERESTS SECTION
%------------------------------------------------------------------------------


\section{Interests}

\renewcommand{\listitemsymbol}{-~} % Changes the symbol used for lists

\cvlistdoubleitem{Trading}{Fishing}
\cvlistdoubleitem{Music Performance}{Long distance running}
\cvlistitem{HIIT}

\comment{
\section{Courses Highlight}
\renewcommand{\listitemsymbol}{-~} % Changes the symbol used for lists
\cvlistdoubleitem{Deterministic O.R. Models}{Stochastic Models for Engineer}
\cvlistdoubleitem{Modern Heuristic}{Integer Programming}
\cvlistdoubleitem{Simulation And Stochastic Models}{Discrete Event Simulation}
\cvlistdoubleitem{Design of Experiment}{Design Optimization}
\cvlistdoubleitem{Probability And Stats}{Data Mining}
\cvlistdoubleitem{Supply Chain Analysis and Design}{Applied Linear Technique}
}


}
\end{document}

